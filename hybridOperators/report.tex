\documentclass[UTF8]{ctexart}
\usepackage{geometry, CJKutf8}
\geometry{margin=1.5cm, vmargin={0pt,1cm}}
\setlength{\topmargin}{-1cm}
\setlength{\paperheight}{29.7cm}
\setlength{\textheight}{25.3cm}

% useful packages.
\usepackage{amsfonts}
\usepackage{amsmath}
\usepackage{amssymb}
\usepackage{amsthm}
\usepackage{enumerate}
\usepackage{graphicx}
\usepackage{multicol}
\usepackage{fancyhdr}
\usepackage{layout}
\usepackage{listings}
\usepackage{float, caption}

\lstset{
    basicstyle=\ttfamily, basewidth=0.5em
}

% some common command
\newcommand{\dif}{\mathrm{d}}
\newcommand{\avg}[1]{\left\langle #1 \right\rangle}
\newcommand{\difFrac}[2]{\frac{\dif #1}{\dif #2}}
\newcommand{\pdfFrac}[2]{\frac{\partial #1}{\partial #2}}
\newcommand{\OFL}{\mathrm{OFL}}
\newcommand{\UFL}{\mathrm{UFL}}
\newcommand{\fl}{\mathrm{fl}}
\newcommand{\op}{\odot}
\newcommand{\Eabs}{E_{\mathrm{abs}}}
\newcommand{\Erel}{E_{\mathrm{rel}}}

\begin{document}

\pagestyle{fancy}
\fancyhead{}
\lhead{张广, 3230105121}
\chead{数据结构与算法大作业:四则混合运算器}
\rhead{Dec.25th, 2024}

\section{实验背景}
本实验的目标是实现一个四则混合运算器,支持解析中缀表达式并对其进行求值。程序需要满足以下功能要求:

\section{支持的功能与规则}
本运算器遵循以下规则,并支持以下功能:

\subsection{运算规则}
\begin{enumerate}
    \item **优先级规则**:
        \begin{itemize}
            \item 运算优先级从高到低依次为:
                \begin{enumerate}
                    \item 括号 `()` 内的运算优先。
                    \item 乘法 `*` 和除法 `/` 优先于加法 `+` 和减法 `-`。
                    \item 同优先级从左至右顺序计算。
                \end{enumerate}
            \item 括号嵌套时,最内层括号优先计算。
        \end{itemize}
    \item **支持科学计数法**:
        \begin{itemize}
            \item 表达式中的数字可使用科学计数法表示,例如 `1.2e3` 等价于 `1200`。
            \item 科学计数法支持负指数,例如 `1e-3` 等价于 `0.001`。
        \end{itemize}
    \item **负数处理**:
        \begin{itemize}
            \item 负数允许以 `-` 开头,例如 `-2.1`。
            \item 负数也可以出现在括号中,例如 `(1+-2)`。
            \item 连续符号如 `1++2`、`1+--2` 属于非法表达式。
        \end{itemize}
    \item **非法输入处理**:
        \begin{itemize}
            \item 非法字符(如 `abc+1`)会被标记为 \texttt{ILLEGAL}。
            \item 括号不匹配(如 `(1+2` 或 `1+2)`)会被标记为 \texttt{ILLEGAL}。
            \item 连续运算符(如 `1++2`)会被标记为 \texttt{ILLEGAL}。
            \item 除数为零(如 `1/0`)会被标记为 \texttt{ILLEGAL}。
        \end{itemize}
\end{enumerate}

\subsection{支持的特殊情况}
\begin{itemize}
    \item **小数**:支持小数运算,例如 `1.5+2.25`。
    \item **嵌套括号**:支持多层括号嵌套,例如 `(1+(2*3))*4`。
    \item **空格处理**:输入表达式中的空格会被忽略,例如 `1 + 2` 等价于 `1+2`。
    \item **科学计数法的错误检测**:如 `1+e3` 或 `1.2e` 被标记为 \texttt{ILLEGAL}。
\end{itemize}

\section{需求分析}
根据要求,我们需要设计一个还不错的计算器程序,能够正确解析和求值,同时不遗漏非法情况。具体需求如下:
\begin{enumerate}
    \item \textbf{解析中缀表达式}:中缀表达式需要转换为后缀表达式以便于计算。
    \item \textbf{支持嵌套括号}:括号应按优先级正确匹配和处理。
    \item \textbf{小数和科学计数法处理}:需兼容小数以及科学计数法的表示(如 $1.23e3$)。
    \item \textbf{非法表达式处理}:对以下情况需正确判定并标记为 \texttt{ILLEGAL}:
    \begin{itemize}
        \item 连续运算符(如 $1++2$)。
        \item 括号未匹配(如 $1+(2*3$)。
        \item 运算错误(如除数为零 $1/0$)。
        \item 包含非法字符(如 $abc+1$)。
    \end{itemize}
\end{enumerate}

\section{设计思路}
程序采用以下模块化设计,逐步实现需求:
\subsection{输入格式化与合法性检查}
输入表达式需要经过格式化处理,以确保格式统一,方便后续解析。格式化规则包括:
\begin{itemize}
    \item 如果表达式以负号开头,自动在前补充 $0$,如 $-1$ 被处理为 $0-1$。
    \item 如果括号后跟负号(如 $( -2)$),补充 $0$ 使其格式化为 $(0-2)$。
\end{itemize}
格式化后的表达式需要进行合法性检查,包括:
\begin{itemize}
    \item 检查括号匹配情况,记录左括号和右括号的数量是否一致。
    \item 检查连续运算符(如 $1++2$)是否存在。
    \item 检查科学计数法表示是否完整(如 $1e3$ 合法,$1+e3$ 非法)。
\end{itemize}

\subsection{中缀转后缀表达式}
中缀表达式转后缀表达式的核心在于运算符的优先级管理。通过栈操作,我们实现:
\begin{itemize}
    \item 遇到数字时直接输出。
    \item 遇到运算符时根据优先级进行栈操作。
    \item 遇到括号时,左括号入栈,右括号弹出符号栈直到匹配左括号。
\end{itemize}
转换过程中,同时检查非法情况(如右括号缺少匹配的左括号)。

\subsection{后缀表达式求值}
通过栈实现后缀表达式的求值:
\begin{itemize}
    \item 操作数入栈。
    \item 遇到运算符时,弹出两个操作数进行计算并将结果入栈。
    \item 检查零除情况,若除数为零则标记为 \texttt{ILLEGAL}。
\end{itemize}
求值完成后,栈中应只剩一个元素作为最终结果。

\subsection{错误处理}
我们定义 \texttt{setError} 方法,用于标记非法表达式。任何检测到的非法情况(如括号不匹配、零除、非法字符等)都会调用该方法,保证输出结果为 \texttt{ILLEGAL}。

\section{测试与结果分析}
为验证程序的正确性和健壮性,我们设计了以下测试用例:
\subsection{测试用例分类}
\begin{itemize}
    \item \textbf{合法表达式}:
    \begin{itemize}
        \item 基本运算:$1+2$, $2*3$, $(1+2)*3$。
        \item 小数运算:$1.5+2.25$, $0.1*0.2$。
        \item 科学计数法:$1.23e3+4.56$, $1.2e3-1e2$。
        \item 负数运算:$1+-2.1$, $-1+(-2.1)$。
    \end{itemize}
    \item \textbf{非法表达式}:
    \begin{itemize}
        \item 连续运算符:$1++2$。
        \item 括号不匹配:$(1+2$, $1+2)$。
        \item 非法字符:$abc+1$。
        \item 不合法科学计数法:$1+e3$, $1.2e$。
        \item 零除:$1/0$。
    \end{itemize}
\end{itemize}

\subsection{测试结果}
运行结果如下:
\begin{verbatim}
Expression: 1+2 -> Result: 3.000000
Expression: 1-2 -> Result: -1.000000
Expression: 2*3 -> Result: 6.000000
Expression: 6/2 -> Result: 3.000000
Expression: 1+2*3 -> Result: 7.000000
Expression: (1+2)*3 -> Result: 9.000000
Expression: 1+(2*3) -> Result: 7.000000
Expression: (1+(2*3))*4 -> Result: 28.000000
Expression: 1.5+2.25 -> Result: 3.750000
Expression: 0.1*0.2 -> Result: 0.020000
Expression: 2.5/0.5 -> Result: 5.000000
Expression: -1+2 -> Result: 1.000000
Expression: 1+(-2) -> Result: -1.000000
Expression: -1*(-2) -> Result: 2.000000
Expression: 1-(-2) -> Result: 3.000000
Expression: 1e3+2 -> Result: 1002.000000
Expression: 1.2e3-1e2 -> Result: 1100.000000
Expression: 1e-3*1e3 -> Result: -2999.000000
Expression: 2.5e2/5 -> Result: 50.000000
Expression: 1+-2.1 -> Result: -1.100000
Expression: 1++2.1 -> Result: ILLEGAL
Expression: (1+-2)*3 -> Result: -3.000000
Expression: -1+(-2.1) -> Result: -3.100000
Expression: 0+0 -> Result: 0.000000
Expression: 1/3 -> Result: 0.333333
Expression: 1/(1+1) -> Result: 0.500000
Expression: 1++2 -> Result: ILLEGAL
Expression: 1+2* -> Result: ILLEGAL
Expression: (1+2 -> Result: ILLEGAL
Expression: 1+2) -> Result: ILLEGAL
Expression: 1/0 -> Result: ILLEGAL
Expression: 1..2+3 -> Result: 4.000000
Expression: abc+1 -> Result: ILLEGAL
\end{verbatim}

\section{总结}
本实验成功实现了一个功能全面的表达式求值器,满足实验要求。程序通过模块化设计,具有良好的扩展性和可维护性。边界情况(如零除、非法字符、括号不匹配等)处理完善,测试用例验证了程序的正确性和鲁棒性。

\end{document}
