\documentclass[UTF8]{ctexart}
\usepackage{geometry, CJKutf8}
\geometry{margin=1.5cm, vmargin={0pt,1cm}}
\setlength{\topmargin}{-1cm}
\setlength{\paperheight}{29.7cm}
\setlength{\textheight}{25.3cm}

% useful packages.
\usepackage{amsfonts}
\usepackage{amsmath}
\usepackage{amssymb}
\usepackage{amsthm}
\usepackage{enumerate}
\usepackage{graphicx}
\usepackage{multicol}
\usepackage{fancyhdr}
\usepackage{layout}
\usepackage{listings}
\usepackage{float, caption}

\lstset{
    basicstyle=\ttfamily, basewidth=0.5em
}

% some common command
\newcommand{\dif}{\mathrm{d}}
\newcommand{\avg}[1]{\left\langle #1 \right\rangle}
\newcommand{\difFrac}[2]{\frac{\dif #1}{\dif #2}}
\newcommand{\pdfFrac}[2]{\frac{\partial #1}{\partial #2}}
\newcommand{\OFL}{\mathrm{OFL}}
\newcommand{\UFL}{\mathrm{UFL}}
\newcommand{\fl}{\mathrm{fl}}
\newcommand{\op}{\odot}
\newcommand{\Eabs}{E_{\mathrm{abs}}}
\newcommand{\Erel}{E_{\mathrm{rel}}}

\pagestyle{fancy}
\fancyhead{}
\lhead{张广, 3230105121}
\chead{数据结构与算法第七次作业}
\rhead{November.28th, 2024}

\begin{document}

\section{引言}
堆排序(HeapSort)是一种基于堆结构的高效排序算法。通过构建大顶堆或小顶堆,并反复执行堆化操作,可以实现时间复杂度为 $O(n \log n)$ 的排序。该报告详细记录了本人堆排序的实现过程、测试方法、性能对比以及算法分析。

\section{算法描述}
\subsection{整体思路}
本人对于堆排序设计的思路如下:
\begin{enumerate}
    \item 将输入数组构建为\textbf{大顶堆},此时最大元素位于堆顶。
    \item 将堆顶元素与堆尾元素交换,将对顶元素(最大元素)放在了数组的末尾,实现了单个元素的正确排序。
    \item 对堆顶元素执行堆化操作(\texttt{siftDown}),恢复堆的性质。
    \item 重复上述步骤,每一次循环可以将单个元素正确排序,而直至堆的未排序长度变为1的时候,就完成了从小到大的排序。
\end{enumerate}

\subsection{必要函数的功能}
本次实现的堆排序主要包括以下两个函数:
\begin{itemize}
    \item \texttt{siftDown}: 用于堆化操作。它从节点i开始,从顶至底进行堆化操作,恢复堆的性质。
    \item \texttt{heapSort}: 堆排序的主函数,首先完成建堆操作,从最后一个非叶结点开始向前遍历至根节点,完成堆化操作;然后交换根节点和最右叶节点,并以根节点为起点从顶至底进行堆化,然后循环这样的过程最终完成堆排序。
\end{itemize}

\subsection{实现细节}
在实现中,我采取了如下策略:
\begin{enumerate}
    \item \textbf{构建大顶堆:} 从最后一个非叶节点开始向上进行堆化操作,时间复杂度为 $O(n)$。
    \item \textbf{堆化操作:} 利用 \texttt{siftDown} 函数对当前堆顶元素进行调整,直到满足堆的性质。
    \item \textbf{原地排序:} 在堆排序中,不需要额外的存储空间,直接在输入数组上完成操作。
\end{enumerate}

\section{测试流程}

\subsection{测试方法}
\begin{enumerate}
    \item 随机生成 4 种长度为 $10^6$ 的测试序列,包括随机序列、有序序列、逆序序列和部分重复序列。
    \item 调用自实现的 \texttt{heapSort} 和标准库的 \texttt{std::sort\_heap()} 对各序列分别进行排序。
    \item 使用 \texttt{chrono} 记录每次排序的时间,并验证排序结果的正确性。
    \item 通过 Valgrind 工具检测内存泄漏,确保程序运行稳定。
\end{enumerate}

\section{性能分析}
\subsection{测试概述}
按要求针对以下四种序列对堆排序进行了测试:
\begin{enumerate}
    \item \textbf{随机序列}:元素为随机生成的整数。
    \item \textbf{有序序列}:元素已经按升序排列。
    \item \textbf{逆序序列}:元素按降序排列。
    \item \textbf{部分重复序列}:包含大量重复元素。
\end{enumerate}

所有序列长度均为 $10^6$,并与标准库的 \texttt{std::sort\_heap()} 进行了性能对比。

\subsection{测试结果}
以下是堆排序与 \texttt{std::sort\_heap()} 的性能测试结果(单位:秒):
\begin{table}[h!]
    \centering
    \begin{tabular}{|c|c|c|}
    \hline
    \textbf{序列类型} & \textbf{堆排序时间 (s)} & \textbf{std::sort\_heap 时间 (s)} \\ \hline
    随机序列        & 0.538                     & 0.310                             \\ \hline
    有序序列        & 0.427                     & 0.249                             \\ \hline
    逆序序列        & 0.457                     & 0.271                             \\ \hline
    部分重复序列    & 0.533                     & 0.336                             \\ \hline
    \end{tabular}
    \caption{堆排序与 std::sort\_heap 的性能对比}
\end{table}

\section{时间复杂度与效率差异分析}
\subsection{时间复杂度分析}
堆排序的时间复杂度分为两部分:
\begin{itemize}
    \item \textbf{构建大顶堆:} 对每个非叶节点执行堆化操作,总复杂度为 $O(n)$。
    \item \textbf{堆化与排序:} 从堆中提取最大元素的时间复杂度为O(logn),共循环n-1轮,故提取最大元素(包括堆化操作)的时间复杂度为$O(nlogn)$。
\end{itemize}

所以总的时间复杂度为$O(n)+O(nlogn)=O(nlogn)$。

\subsection{与 \texttt{std::sort\_heap()} 的效率差异}
从测试结果可以看出,标准库的 \texttt{std::sort\_heap()} 明显快于我实现的 \texttt{heapSort},我个人认为原因如下:
\begin{enumerate}
    \item \textbf{底层优化:} 标准库的实现经过高度优化,可能使用了硬件加速或并行化操作,而我的实现只针对通用性,没有针对性能进行优化。
    \item \textbf{缓存命中率:} 标准库对内存访问模式进行了优化,减少了缓存未命中,而我的实现在交换和堆化操作中可能导致更多的缓存未命中。
    \item \textbf{函数调用开销:} 我的实现中使用了较多的递归或循环,可能带来额外的函数调用开销。
\end{enumerate}


\section{结论}
我实现的堆排序性能较标准库的 \texttt{std::sort\_heap()}有一定的性能差异,但时间复杂度符合理论预期,且能够正确处理多种类型的输入序列,感觉还行(手动狗头)

\end{document}

