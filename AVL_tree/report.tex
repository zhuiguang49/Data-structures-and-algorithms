\documentclass[UTF8]{ctexart}
\usepackage{geometry, CJKutf8}
\geometry{margin=1.5cm, vmargin={0pt,1cm}}
\setlength{\topmargin}{-1cm}
\setlength{\paperheight}{29.7cm}
\setlength{\textheight}{25.3cm}

% useful packages.
\usepackage{amsfonts}
\usepackage{amsmath}
\usepackage{amssymb}
\usepackage{amsthm}
\usepackage{enumerate}
\usepackage{graphicx}
\usepackage{multicol}
\usepackage{fancyhdr}
\usepackage{layout}
\usepackage{listings}
\usepackage{float, caption}

\lstset{
    basicstyle=\ttfamily, basewidth=0.5em
}

% some common command
\newcommand{\dif}{\mathrm{d}}
\newcommand{\avg}[1]{\left\langle #1 \right\rangle}
\newcommand{\difFrac}[2]{\frac{\dif #1}{\dif #2}}
\newcommand{\pdfFrac}[2]{\frac{\partial #1}{\partial #2}}
\newcommand{\OFL}{\mathrm{OFL}}
\newcommand{\UFL}{\mathrm{UFL}}
\newcommand{\fl}{\mathrm{fl}}
\newcommand{\op}{\odot}
\newcommand{\Eabs}{E_{\mathrm{abs}}}
\newcommand{\Erel}{E_{\mathrm{rel}}}

\begin{document}

\pagestyle{fancy}
\fancyhead{}
\lhead{张广, 3230105121}
\chead{数据结构与算法第六次作业}
\rhead{November.9th, 2024}



\section{实现原理}

此报告叙述了AVL树删除节点操作的实现方法,我们在执行删除操作时不仅需要保持二叉搜索树的性质,并且还要按AVL树的方式保持平衡。

\subsection{删除策略}

我们采用递归实现删除操作,有以下三种情况:

\begin{enumerate}
    \item 待删除节点为叶节点:直接删除
    \item 待删除节点有一个子节点:用子节点替换当前节点
    \item 待删除节点有两个子节点:用右子树中的最小值替换当前节点,并删除最小值节点
\end{enumerate}

\subsection{平衡维护}

删除节点后,我们自底向上更新节点高度以保持AVL树的平衡。主要步骤如下:

\begin{enumerate}
    \item 更新节点高度:$height = max(leftHeight, rightHeight) + 1$
    \item 计算平衡因子:$balance = leftHeight - rightHeight$
    \item 根据平衡因子进行旋转操作:
    \begin{itemize}
        \item 左左情况(balance > 1):右旋
        \item 左右情况(balance > 1):先左旋后右旋
        \item 右右情况(balance < -1):左旋
        \item 右左情况(balance < -1):先右旋后左旋
    \end{itemize}
\end{enumerate}

\section{关键代码实现}

删除操作的核心在于平衡的维护。每次删除节点后,我们都需要更新受影响路径上的节点高度,并检查是否需要重新平衡。
重平衡过程的实现步骤如下:

\begin{itemize}
    \item 调用updateHeight更新节点高度
    \item 计算平衡因子
    \item 根据平衡因子的值确定是否需要旋转以及使用哪种旋转方式
    \item 旋转完成后重新更新受影响节点的高度
\end{itemize}

为了提高效率,本实现在递归返回过程中进行平衡维护,这样可以保证每个节点最多只需要调整一次。同时我们按照要求使用移动而非复制来处理节点值的替换,减少了不必要的对象拷贝。


\end{document}

